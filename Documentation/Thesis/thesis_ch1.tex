\documentclass[12pt]{report}
\author{Nicholas Berezny}
\title{Thesis Outline}
\usepackage{cite}
\usepackage{fullpage}
\usepackage{gensymb}
\usepackage{}
\renewcommand{\baselinestretch}{1.5} 

\usepackage{amsmath}
\usepackage{graphicx}
\graphicspath{ {pics/} }
\usepackage{svg}

\begin{document}
\maketitle
\newpage

\chapter{Literature Review and Preliminary Research}

\section{Stroke and Rehabilitation}

Stroke is a cerebrovascular disease which effects approximately 62,000 Canadians and 795,000 Americans each year \cite{Benjamin2018,Hebert2016}. It is among the leading casuses of adult disability \cite{Ewart2003}, with some form of motor impairment effecting 80\% of patients, which can result in limited mobility and muscle control \cite{Langhorne2009}. Ischaemic stroke is the most common form of stroke, accounting for around 80\% of all cases \cite{Rey2008}. It is caused by blood vessel occlusion, either directly in the brain (thrombotic), or due to the migration of a clot formed somewhere else in the body (embolic). The subsequent lack of oxygen causes the death of brain tissue and neurological deficits \cite{Prabhakaran2015}, which manifest as functional impairments in activities associated with the affected brain region. Stroke complications can include hemiparesis (weakness on one side of the body), muscle spacitity, loss of motor control, and loss of dexterity, all of which can effect the patient's ability to perform activities of daily living (ADL's) and compromise their autonomy. In addition to medical intervention in the form of pharmaceuticals and surgery, the stroke patient may require rehabilitation to reduce functional impairment \cite{Stroke}. It is imperative to administer treatment during the acute phase of the stroke (\textit{i.e.} within days of the stroke) \cite{Prabhakaran2015}.

%----
%Approximately 62,000 Canadians suffer from stroke every year, with over 10\% of victims requiring in-patient rehabilitation \cite{Benjamin2018,Hebert2016}. It is among the leading casuses of adult disability \cite{Ewart2003}. Motor impairment effects around 80\% of patients, which can result in limited mobilited and muscle control \cite{Langhorne2009}. The scale of stroke care and rehabilitation is significant, with ... Improving stroke rehabilitation would benefit the stroke victims by improving their recovery and thus their quality of life, and would also have significant economic benefits, with one study estimating up to \$682 million dollars annually in Canada alone, due to savings due to earlier discharges, as well as increased productivity of the stroke patients post-treatment \cite{Krueger2012}. 
%----

%Ischaemic stroke is the most common form of stroke, accounting for around 80\% of all cases \cite{Rey2008}. It is caused by blood vessel occlusion, either directly in the brain (thrombotic), or due to the migration of a clot formed somewhere else in the body (embolic). The subsequent lack of oxygen in the affected area of the brain causes the death of brain tissue and neurological deficits \cite{Prabhakaran2015}, which manifest as functional impairments in the patient in activities associated with the affected brain region. Stroke complications can include hemiparesis (weakness on one side of the body), muscle spacitity, loss of motor control, loss of dexterity, etc... In addition to medical intervention in the form of pharmaceuticals and surgery, the stroke patient may require rehabilitation to reduce functional impairment and, ideally, to recover full autonomy \cite{Stroke}.

Older treatment techniques relied on compensatory measures to assist stroke patients in regaining autonomy \cite{Dobkin2004}. Using rehabilitation to restore autonomy began to be emphasized once the concept of neural plasticity became more widely accepted. Neural plasticity is the process by which new connections are formed in the central nervous system in response to experience or activities performed by the subject \cite{Warraich2010}. When a stroke patient repetitively performs motor tasks, new neural connections in the brain are formed which take on the role of the regions of the brain which were damaged. This allows the patient to re-learn tasks such as walking by going through intensive rehabilitation. 

%Rehabilitation of lost abilities began to be emphasized over compensatory measures due to the discovery of neural plasticity and its role in the process of learning new tasks. The basis of this neural plasticity is the formation of new neural connection in the brain in response to a repeated movement or activity. In the case of stroke, this allows new neurons to take on the role of damaged regions in the brain, allowing the patient to effectively re-learn tasks such as walking. 

Stroke rehabilitation encompasses a broad range of therapeutic activities, typically carried out by occupational- and physiotherapists. Physiotherapists (PT's) tend to focus on gross motor movements in the upper and lower limbs, while occupational therapists (OT's) focus on fine motor control, dexterity, and specific ADL's. These two goals are inextricably linked, and so both PT's and OT's form a team which works together to improve patient outcomes. The rehabilitation carried out by both OT's and PT's involves the repetition of a target movement or task. Depending on the abilities of the patient, the therapist may provide support or assistance. For example, lower-limb rehabilitation begins with simple legs movement performed in bed, such as knee flexion/extension, leg lifts, hip abductions, \textit{etc} [include pictures]. The therapist can either assist the patients leg through the movement, or allow patients to perform the movement on their own, or even provide light resistance if the patient is more advanced. Next, the patient will practice sitting upright, going from sit to stand, and  practice bearing weight and balancing while standing. Eventually the patients will be ready for more complex tasks with assistance from the therapist, including walking, heel strike, and stair climbing. The patient may advance to the point where they can walk without therapist assistance, using either parallel bars, a walker, or potentially without any assistance. The stages of lower-limb rehabilitation are summarized below:

\begin{enumerate}
	\item Bed-bound exercises with assistance
	\item Sit-to-stand and weight bearing
	\item Assisted gait therapy (with therapist moving the leg through the gait trajectory)
	\item Other out-of-bed activities, such as heel strike, stair climbing, etc.
	\item Walking without therapist assistance (using parallel bars or a walker)
\end{enumerate}



\subsection{Rehabilitation Guidelines}

Canadian stroke best practices recommend that therapy involves the repetition of intensive, task-specific activities in a complex and stimulating environment \cite{Hebert2016}. Lower-limb rehabilitation should be goal-orientated towards selected tasks such as walking and sit-to-stand. Robotics, virtual reality, and biofeedback are all mentioned as potential adjuncts, but are not recommended to replace traditional therapy. It is also recommended that patient's should receive rehabilitation as early as possible once they are deemed medically ready. Early mobilisation is recommended (within 24-48 hours of stroke, \cite{Casaubon2016}, but not before 24 hours due to findings that this actually reduces functional outcomes, \cite{AVERTTrialCollaborationgroup2015}). Active participation on the part of the patient is also imperative [4], as it has been found that this facilitates neural plasticity. This requires that the patient be engaged in the therapy (hence the need for stimulating environments), and also requires the therapist to ensures that the rehabilitation is not passive.

Therapy dosage also has an effect on functional outcomes. For example, one study found positive correlation between time scheduled for therapy and most functional outcomes \cite{Lohse2014}. Another study sought to determine if current doses of rehabilitation are enough to trigger functional improvements by comparing with animal models \cite{Lang2009}, and found that the amount of rehabilitation they observed was inadequate. Another study found that stroke inpatients spend on average 13\% engaged in physical activities and 5.2\% of their time with a therapist \cite{Bernhardt2004}. Other studies confirm that stroke patients spend a significant amount of time in or sitting near their bed \cite{King2011}. These results indicate the importance of scheduling time specifically for rehabilitation for stroke inpatients. 

Other factors may restrict the PT's and OT's ability to accomplish the recommended best practices. The intensity and duration of the therapy is limited both by the endurance of the patient and of the therapist, as the physical burden of manually assisting the patient may be high \cite{Colombo}. Furthermore, providing task-specific and stimulating therapy is difficult, particularly when doing bed-bound exercises which are not easily relatable to ADL's. One study found that therapy dosage may be affected by understaffed units, resulting in less rehabilitation time with therapists \cite{Mchugh2013}.


%-----
%Guidelines stipulated in \cite{Hebert2016} express the need for prompt lower-limb rehabilitation, beginning ideally within days of the stroke. This is denoted the ``acute'' phase, as opposed to the ``chronic'' phase which refers to longer-term rehabilitation. The importance of acute stroke rehabilitation has been emphasized in the above guidelines and others like it as a way to improve patient outcomes. This is attributed in part to the brain's greater receptivity to changes through neural plasticity, a mechanism triggered through rehabilitation. However, starting therapy earlier may be more difficult, given that the complications are generally more pronounced. Guidelines also recommend that therapy is intensive, task-specific, and engaging \cite{Hebert2016}.  The intensity and duration of therapy may be limited by the endurance of the patient, as well as the physical burden that manually assistance places on the therapist. The frequency of therapy may also be limited by the busy schedules of PT's and OT's. 



 %Providing physical support and assistance to patient's can be burdensome for the therapist, especially when transferring patients out of bed and standing them up. The repetitive nature of rehabilitation can also be physically demanding. 


%The rehabilitation carried out by both OT's and PT's involves the repetition of a target movement or task. Depending on the abilities of the patient, the therapist may provide support or assistance. For example, typical lower-limb rehabilitation begins with simple legs movement performed in bed, such as knee flexion/extension, leg lifts, hip abductions, \textit{etc}. The therapist can either assist the patients leg through the movement, or allow patients to perform the movement on their own, or even provide light resistance if the patient is more advanced. Providing physical support and assistance to patient's can be burdensome for the therapist, especially when transferring patients out of bed and standing them up. The repetitive nature of rehabilitation can also be physically demanding. 

%A stroke patient's progress is monitored through tests, surveys, and measures...

%Inpatient's at the hospital may be discharged without having fully recovered from their impairments. They may seek outpatient clinics for further, chronic stroke rehabilitation. Many never fully recover the ability to perform all of their necessary ADL's (ref). This fact, coupled with the increasing rates of stroke, has spurred a variety of active research into improvements to current stroke intervention, and novel ways of administering rehabilitation. 

%% OLD 
%It is recommended that patients receive three hours a day of therapy, with more therapy generally leading to better outcomes. Exercises should be task-specific, meaningful, and related to activities of daily living. Giving the patient chances to repeat the exercises (under supervision) is also recommended.  The guidelines also explicitly mention both robotic devices and virtual reality training as potential tools for rehabilitation, although they should be used in addition to traditional therapy and not as replacements.  


\section{Robotic Rehabilitation}

	Rehabilitation robots have been introduced to improve the stroke rehabilitation process by automating the physical assistance usually supplied by therapists. Rehabilitation requires the repetition of a precise motion, a task well-suited to robotics. Over the past few decades, a variety of robotic platforms have been developed for many different therapeutic activities, from lower-limb gait training to fine motor skills in the hand. There are many other potential benefits for using robotics for rehabilitation, including the following.
\begin{itemize}
	\item \textbf{Relieving Physical Burden}: Using robots to provide the physical support and assistance required for stroke therapy will relieve the physical burden from therapists, removing the chance of injury or fatigue.
	\item \textbf{Increased Intensity and Duration}: Both the intensity and duration of a therapy session may be limited when relying on a therapist to provide physical assistance. With a robot, duration and intensity would only be limited by the patient. 
	\item \textbf{Increased Dosage}: Robots could also increase the amount of rehabilitation given to patients. For example, a single therapist could supervise several patient's each using a robotic device, therefore increasing the number of patients a therapist can manage.
	\item \textbf{Providing Performance Measures}: Robots record a number of metrics not available when using traditional therapy, such as velocity, force, trajectory error, and effort. Tracking these measures could allow the therapist to better administrate therapy and track patient improvement. 
	\item \textbf{Virtual Environments}: Patient engagement could be increased by incorporating virtual reality, games, and other forms of visual feedback. The robot would serve as the control input to the environment. The relevancy of the therapy could also be improved by simulating ADL's. Finally, haptic feedback could further immerse the patient in the virtual environment by rendering virtual forces to the patient (see section \ref{Sec:VR}).
\end{itemize}
	

	Rehabilitation robots can be broadly categorized as targeting the lower-limb or the upper-limb. Upper-limb robots tend to be lighter weight since they encounter much smaller loads, while lower-limb devices are heavier as they need to support either the weight of the leg or of the entire body. Lower-limb robots can be further categorized as either targeting standing activities such as treadmill walking, or lying/sitting activities \cite{Calabro2016}. A final subdivision exists between endplate based devices and exoskeletal devices. Endplate-based devices interact with the patient at a single point, usually the end of the limb (the hand or foot), and only apply forces through this point. Exoskeletal devices attach to the user at multiple points, offering greater control over the overall motion of the limb, although they are often more restrictive, more complex, and potentially less safe \cite{Chang2013}. One study found evidence that endplate-based devices may actually yield better outcomes than exoskeletal devices, although the comparison was done via meta-analysis of many different studies with significantly different methodologies \cite{Mehrholz2012} . 
	
	Rehabilitation robots have not yet been fully endorsed by stroke best practices. For example, Canadian best practices list robots as a potential benefit, but do not recommend they replace traditional therapy \cite{Hebert2016}. This is largely due to the limited evidence for outcomes of robotic rehabilitation. Many studies are small scale case studies, and the 
few large clinical trials have found conflicting results. One study following 63 patients compared using the Lokomat (see Sec. \ref{Sec:Overground}) to conventional therapy, and found that the robot was less effective \cite{Hidler2008}. There are many challenges in studying the efficacy of robots in a clinical setting. It is unethical to replace traditional therapy with an unproven method, so typically extra therapy is added to the subject's regimen either from a robot or from a therapist (control). Since most robots are designed to replace time spent in conventional therapy, the robot must consistently be shown to be at least as effective as a therapist at delivering rehabilitation. Furthermore, since most robotic devices are substantially different, results from the study of one lower-limb robot may not apply to another robot. All of this makes ''proving`` the efficacy of a rehabilitation robot difficult. That said, there have been many different robotic devices successfully tested, with some even being available commercially. 
	
	
	\subsection{Robots for Overground Walking} \label{Sec:Overground}

	One of the most important activities for lower-limb rehabilitation is gait training, wherein the patient walks with assistance from the therapist. Rehabilitation robots were introduced to provide this assistance by attaching to the patient's leg and guiding them through the correct gait trajectories. This can be accomplished bv an exoskeletal orthosis attached around the knee joint, or by and endplate device which supports and guides the feet. 
	
	The Lokomat, ALEX \cite{Banala2007}, LOPES \cite{Veneman2007} and ReoAmbulator are examples exoskeletal gait trainers. Each comprise of a powered exoskeleton which can assist the leg through gait trajectories. They all work in conjunction with body-weight supported treadmill training (BWSTT), using a harness to support the patient and a treadmill for walking. The Lokomat and LOPES use impedance control to deliver programmable levels of assistance to the user. ALEX uses a force-field scheme to create an ''assist-as-needed`` controller which ensures that the user contributes to the movement.
	
	The Lokomat is one of more widely researched platforms. One recent study found that the Lokomat was more effective than conventional gait therapy with regards to certain outcome measures \cite{Nam2017}, although another found that it was less effective \cite{Hidler2008}. Despite conflicting results, the Lokomat continues to show promise. It is commercially available -- 872 devices can be found worldwide according to their website (as of 2019).
			
	\begin{figure*}[h] 
		\centering
		\includegraphics[width=0.75\linewidth]{Lokomat}
		\caption{The Lokomat (by Hocoma), an exoskeletal robot}
		\label{fig:Lokomat}
	\end{figure*}
	
	
	The Haptic Walker \cite{Schmidt2005}, G-EO \cite{Hesse2010}, and Gait Trainer \cite{Hesse} are examples of end-plate based rehabilitation robots. Both use actuated platforms under the feet to move the patient through gait trajectories, and a harness to support the patient. The Haptic Walker can perform walking simulations and more complex situations like stumbling or climbing stairs using a 3 DOF system. As can be seen in Figure \ref{fig:Hapticwalker}, the system is much larger than the exoskeletal systems. The GE-O system is similar in that it can simulate walking and stair climbing, with one study finding that electromyography (EMG) readings were similar between simulated walking with the robot and real walking \cite{Hesse2010}. 
	
	\begin{figure*}[h] 
		\centering
		\includegraphics[width=0.75\linewidth]{Hapticwalker}
		\caption{The Hapticwalker, an endplate based robot}
		\label{fig:Hapticwalker}
	\end{figure*}
	
	Overground walking robots show promise for improving gait rehabilitation. However, these only target a single type of lower-limb rehabilitation. Patients must be somewhat mobile (enough to leave their beds and their rooms) to access these devices, and the complexity and setup require that a therapist be present. Hence these robots are designed to replace time spent on conventional therapy, which makes proving consistent performance difficult as was previously mentioned.  
	
	\subsection{Robots for Sitting or Bed-bound Therapy}
	
	Robots for assisting in activities while the patient sits or lies down are less common in the literature. They can be used on acute patients who are less mobile. The MotionMaker, Lambda, and ViGRR are some examples. 
	
	The MotionMaker consists of a reclinable chair and two orthoses, which attach to the leg only at the foot \cite{Schmitt2004}. It uses a regulator with force feedback to control the exercise. It is designed to be used in conjunction with functional electrical stimulation for spinal cord injury patients, but it's application extends to lower-limb rehabilitation in general. 
	
	\begin{figure*}[h] 
		\centering
		\includegraphics[width=0.75\linewidth]{Motionmaker}
		\caption{The MotionMaker}
		\label{fig:Motionmaker}
	\end{figure*}
	
	The Lambda is robot developed for general lower-limb rehabilitation and for fitness purposesd \cite{Bouri2009}. It uses parallel linkages with a total of 3 DOFs to achieve a variety of motions. An adjustable seat allows the user to recline as needed. Recent developments include the addition of virtual reality and games to engage the patient, enhanced feedback regarding the patients progress, and multiple types of exercises. 
	
	\begin{figure*}[t] 
		\centering
		\includegraphics[width=0.75\linewidth]{Lambda}
		\caption{The Lambda}
		\label{fig:Lambda}
	\end{figure*}
	
	
	The precursor to this project is the Virtual Gait Rehabilitation Robot (ViGRR), a device developed at Carleton University's Advanced Biomechatronics and Locomotion Laboratory \cite{Chisholm2014, Chisholm2010}. It is a 4-DOF end-plate based robot which can assist a reclined patient move through trajectories in the sagittal plane. The user's foot interfaces with a footplate which is magnetically attached to the robot. The magnet can be released if triggered by safety system, including high force, high velocity, or pressing an emergency stop button. An admittance controller is used to provide assistance-as-needed, roughly in proportion to the trajectory error. These desired trajectories can include gait trajectories, linear motion, circular motion, or any other path in the sagittal plane. To increase user engagement, a virtual environment was created which can display visual feedback regarding the desired trajectory, or games such as pushing a block or brick breaker. User's can be further immersed in the virtual environment through haptic feedback. For example, when pushing on virtual wall, the robot will apply a force to the user's foot emulating the wall reaction force. 
	
	
	\begin{figure*}[t] 
		\centering
		\includegraphics[width=0.75\linewidth]{Vigrr}
		\caption{ViGRR}
		\label{fig:vigrr}
	\end{figure*}	
	
	
	%The motivation to create a new device is twofold: creating a lower-powered device suitable for acute stroke rehabilitation, and a smaller and more light-weight device suitable for the hospital environment. 

	
	
	\subsection{Controls} 
	
	The type of controller used is especially important for devices interacting with a human (or any unknown environment), as it will determine the interaction behaviour and stability. The behaviour of the interaction should be optimized to ensure the patient engages in meaningful therapy driving neural plasticity and recovery. Position control through the desired trajectory is not adequate, as the robot will effectively force the limb through the motion, thus becoming a passive motion machine with no necessary engagement from the patient. Neural plasticity requires engagement, \textit{i.e.} the patient needs to contribute to some extent to the movement of their limbs. Therefore, rehabilitation robotics should only give some assistance to the patient, instead of leading the motion. 
	
	Three popular control laws for rehabilitation robotics are listed in \cite{Meng2015}:
	
\begin{itemize}

	\item \textbf{Impedance and admittance} control were introduced by Hogan in \cite{Hogan1985}, and have been the foundation of most interaction controllers since. These controllers determine the dynamic interaction between robot and environment instead of regulating position or force. In this way, an impedance controller can be used to give a programmable amount of assistance to the user through a trajectory, which can be varied by changing the parameters so that an optimal amount of assistance can be obtained for each individual. The desired impedance or admittance is defined in terms of a mass-spring-damper system. Therefore, changing the spring gain will change the level of assistance. The damper gain can be used to resist motion if the patient is more advanced or if strength training is the goal.
	
	\item \textbf{EMG-based} control uses electromyography (EMG) to measure muscle activity in the patient. The robot then only provides assistance when the intent to move is detected (\textit{i.e.} the patient flexes their muscle). This ensure that the patient is actively engaged in the therapy. 
	
	\item \textbf{Adaptive} control here refers to a controls which changes its behaviour over time based on some predefined law. For example, the amount of assistance supplied by an admittance controller can be changed based on patient performance. Some robots use information from movements in the patient's healthy limb to calculate the optimal control path for the affected limb. Others use assist-as-needed control schemes to only provide assistance when the patient requires it. Artificial Neural Networks have been used to determine when the assistance is needed in robots such as the ALEX and LOPES.
	
\end{itemize}
		
	\subsection{Virtual Environments} \label{Sec:VR}
	
	Virtual environments or virtual reality refer to digitally created simulations or visualizations, typically displayed through headsets or on computer monitors. It combines visuals, auditory stimulation, and haptic feedback to create an immersive experience. Virtual reality has been introduced into rehabilitation for two primary reasons \cite{Laver2015}:
	\begin{itemize}
		\item \textbf{Increase Relevancy}: simulate ADL's and real world activities to introduce context into the exercise, and better prepare the patient for these activities outside of the hospital environment.
		\item \textbf{Increase Engagement}: use games and other visuals to make the therapy more exciting and engaging, encouraging more active participation, higher intensity, and higher doses of activity. 
	\end{itemize}

	Virtual reality can be coupled with rehabilitation robotics to further increase immersion. The robot can be used as a control input to the virtual environment, and the robot can in turn deliver haptic feedback to the user. Haptic feedback allows users to interact directly with the virtual world, for example when walking, the ground reaction force can be applied to the user's foot. This increases the realism of ADL simulations. 
	
	A metastudy found evidence that VR may improve functional outcomes over conventional therapy, but admits that most studies are small and prone to bias \cite{Laver2015}. One platform used an upper-limb exoskeletal device to apply assistive forces  \cite{Patel2015}. Another study used a Nintendo Wii gaming system to promote therapy, and found it to be a feasible and safe method  \cite{Saposnik2010}. Another study used the Phantom Omni, a pen attached to a robotic arm that can deliver haptic feedback from a virtual environment \cite{Jiang2017}. A serious gaming approach is used in \cite{SociedadeBrasileiradeInformaticaemSaude2014}, where the Unity3D engine was used to create a virtual world and a MS Kinetic sensor to detect the user's motions. 
	
	
\section{Interviews and Shadowing}

%------------------------------------------------------------------------
%-shadowing notes
%
%-------------------------------------------------------------------------
	
	The rise of the human-centred design paradigm underscores the importance of involving end users in the design process, especially when the designers lack expertise in the field of interest. This is certainly the case for this project as none of the engineers involved have ever delivered stroke therapy. As such, the first step of the design process was to consult the end user and collect qualitative data and advice. In the case of rehabilitation robots, there are two types of end-users: the stroke patient using the device, and the therapists who decide when and how to incorporate the device into their rehabilitation regimen. The therapist, however, has greater expertise with stroke rehabilitation in general, and will ultimately be the one who decides the device's fate on the stroke ward. Therefore, it was decided to interview and shadow several PT's and OT's concerning the direction of the project. Patient's were not consulted, due to their lack of expertise in stroke rehabilitation, although their input will be vital in future evaluation experiments. 
	
	Fieldwork was conducted at the Civic campus of the Ottawa Hospital (TOH), in cooperation with Dr. Dariush Dowlatshahi, and with ethics approval from the Ottawa Health Science Network Research Ethics Board (OHSN-REB). Three PT's and two OT's were recruited for two days of interviews and shadowing. Inclusion criteria included experience working on the stroke ward delivering rehabilitation to stroke patients. Not enough data was collected to perform any meaningful statistic -- instead, key insights guided the fundamental concept of the device, and information was gathered not otherwise available through a literature review. Some topics of interest included:
	
	\begin{itemize}
		\item Typical acute, bed-bound rehabilitation 
		\item Patient Engagement -- how the therapists keep patients motivated and interactive
		\item The hospital environment
		\item The therapists' views on rehabilitation robotics 
	\end{itemize}

	
	\subsection{Typical Rehabilitation}
	
	Bed-bound lower-limb rehabilitation is one of the first therapeutic activities administered at the stroke ward, and it continues to be used even after patients are mobile. Bed-bound exercises tend to be simple movements, including knee flexion/extension, leg lifts, hip adduction/abduction, and ankle rolls. All exercises involve the therapist manually manipulating the patient's leg, providing assistance to whatever extent the therapist deems necessary. If the patients are advanced, the therapist may choose to resist motion instead of assisting. 
	
	Parameters from typical rehabilitation will help determine the requirements of the new robotic system. Therapists estimated that they exert a maximum of 10 - 15 lbs of force to the leg. The range of motion of patients varied, but many could go through the full range of the leg. The exercises were done slowly at a consistent speed. The role of the therapist can be summarized as providing support to leg and providing assistance through the motion. 
	
	Some patient's are capable of performing bed-bound exercises independent of the therapist. Performing independent practice can increase total rehabilitation dosage and is recommended by stroke guidelines \cite{Hebert2016}. However, some patients who are either at an earlier stage of recovery or have suffered more severe strokes are incapable of performing bed-bound exercises without assistance, and so their total dosage is limited to time spent with the therapist. 
	
	\subsection{Patient Engagement} 
	 It is important for the patient to be actively engaged in the exercise. This is accomplished through verbal encouragement, and by requiring the patient to initiate the movement. Therapists also used physical cues to guide the patient, for example by tapping the leg if it needs adjusting. Many of the patients required verbal or physical engagement throughout the process. This is in part due to a phenomenon known as neglect, wherein the stroke victim has trouble focusing on the effected side of the body. Sometimes, pictures and posters (\textit{e.g.}of cats and dogs) were posted in front of the exercise machines in the therapy room. Another common practice is to switch to a new exercise if the patient is showing signs of losing focus. 
	 
	 \subsection{Hospital Environment}
	Space in the hospital is limited. Many rooms house four patients, and are often crowded by hospital staff, family, and equipment. The beds themselves vary in model and size, but all have some common characteristics: movable guards on the side, removable baseboard, and a tiltable frame controlled from a panel. There appear to only be outlets behind the beds near the floor, but this will again vary from room to room. 
	There is also a therapy room where more advanced patients go to practice sit-to-stand, walking, stair-climbing, etc. This room is more spacious and also includes beds (albeit simpler beds that cannot be tilted and that do no have guards). 
	
	\subsection{Therapy Schedule}
	
	Therapists generally have a high workload and so must carefully schedule time with patients. At this particular hospital, patients received on average 30 minutes every other day with a physiotherapist for lower-limb rehabilitation. Most patients were fatigued by the end of the session, and so increasing the duration beyond 30 minutes is not feasible. However, the therapists recognized that the patients could benefit from more sessions throughout the week. 
	
	\subsection{Other Comments and Concerns}
	
	Most therapists had concerns over safety. They recognized that the robot could potential apply forces which could harm the patient. One point of concern was rotation of the leg out of position such that the force was applied incorrectly. For example, during the knee flexion/extension exercise, if the leg were to rotate out of the sagittal plane, the robot could apply a torque about the hip which would bend the leg further and potentially injure the patient. Therapists suggested having multiple support points along the leg to keep the leg in position. 
	
	Other comments included the need for comfort, particularly using comfortable material where the robot and human interact. They also mentioned the use of tactile methods of engaging the patient (\textit{e.g.} tapping or vibrations applied to the leg). Some indicated that combining upper and lower-limb activities could be beneficial (\textit{e.g.} by using a joystick with the robot). 
	
	
\section{Device Concept} 

With ViGRR, our goal was to design a lower-limb rehabilitation robot targeting acute patients by creating a device compatible with a patient who is lying or reclined in a hospital bed. ViGRR accomplishes this, and has been validated on healthy subjects. However, some aspects of ViGRR prevent it from being used easily in the hospital, including its high weight, high power, and size. The next stage in the project is to repackage ViGRR into a more accessible and light-weight design. 

This new prototype should align with ViGRR's fundamental concept: a lower-limb robot capable of assisting bed-bound acute stroke patients with rehabilitation. The device should be a viable tool for therapists within the hospital to use to increase functional outcomes for their patients. It should replicate the role of the therapist when providing assistance for bed bound exercises. There are three primary roles:

\begin{enumerate}
	\item Support Leg 
	\item Provide Assistance through the motion
	\item Engage the patient
\end{enumerate}

Supporting the leg will be accomplished through the physical structure of the device. Assistance will be provided using force feedback and interaction controllers which can ensure that the patient is contributing to some extent. The patient will be engaged using virtual environments and haptic feedback. 

Bed-bound exercises include a variety of simple movements, such as leg lifts, knee extensions, and hip abductions. Since this device is a proof-of-concept, it was decided that we would initially target a single exercise, evaluate its performance in the hospital, and then extend to target other exercises by adding degrees-of-freedom. We chose the knee flexion/extension exercise for the following reasons:

\begin{itemize}
	\item It is a commonly used bed-bound exercise
	\item It is important due to its relation to the sit-to-stand activity, which is the first stage of getting the patient mobile
\end{itemize}
 
The knee flexion/extension exercise requires just a single horizontal degree-of-freedom. The therapist uses one hand to support the leg under the knee, and the other hand to support the foot and apply force through the heel. They slowly move the leg back and forth, encouraging the patient to push. They are careful to keep the leg aligned in the sagittal plane. 

There are several use cases for the device. The therapist could use it during therapy instead of manually moving the legs. This would relieve them of the physical burden, the virtual environments may encourage the patient to participate more, and the data collected could be used to track progress. The therapist could use multiple devices simultaneously to provide care to multiple patients. This would allow them to see more patients and thus increase therapy dosage. Finally, since the device replicates the role of the therapist, the device could be used separately from the therapist, under the supervision of other hospital staff. This would encourage patient's to practice independently, thereby increasing therapy dosage. 

In addition to these requirements, the device should be usable within a hospital environment. The device should be safe such that the patient is not harmed, and should also be perceived as safe so that the therapists and patients are willing to use it. The device should fit on a hospital bed, and should be lightweight and small enough to be manageable within a hospital room. The set-up of the device should be quick and easy.



%---
%We set out to design a simple lower-limb rehabilitation for acute, bed-bound stroke patients. We chose to design for the knee flexion-extension exercise because it is most related to sit-to-stand, which is the most important activity for bed-bound patients to recover as it allows the patient to get up and begin gait rehabilitation. The robotic configuration best suited to this exercise is a one degree of freedom linear actuator. 


%Stroke rehabilitation best practices were outlined in the previous section. A few key recommendations highlight the potential benefits of introducing a simple-to-use robotic device into the rehabilitation arsenal. The robotic device would primarily be used to offer additional rehabilitation time to patients under the supervision of a nurse or other responsible person. The gaming aspect of the system would ensure the patient is engaged in the exercise, and also offer feedback so that the patients movements can be regulated even in the absence of a therapist. The games and visualizations could also be tailored to simulate ADL's. Therefore, a robotic device could theoretically boost outcomes by covering the following recommendations:



\section{Contributions and Outline}

A one DOF robot was designed and created, consisting of a DC motor, force sensor, belt drive, footplate, frame, and a number of smaller components. A printed circuit board (PCB) was created to route signals, do preprocessing, and distribute power. System software was implemented on a real-time linux system. A controller was created in C, which can operate the device using admittance control, haptic feedback, a basic physics engine, communication with a UI server, and data logging. Real time functionality is ensured using POSIX standards for memory locking, multi-threading, etc. The UI is implemented as website using the React framework, and some simple games are created using the THREE.js 3D web graphics library. Preliminary experiments are run, including functionality tests and healthy subject testing. 

\begin{itemize}

\item Chapter 2 covers the design of the device. This includes preliminary design choices, the design of the physical structure, the actuation and sensors, the design of the electronics (power distribution, signal routing and preprocessing), and the chosen computer hardware. 

\item Chapter 3 covers the controller and its implementation in the software, including the basic admittance control, and the haptic feedback with physics engine. The design of the user interface (UI) and some basic games/visualisations are also discussed. The communication and the overall software design within real-time Linux is explained, including the how the modular design will allow it to be easily adapted to future design iterations. 

\item Chapter 4 covers testing and experiments, including functional testing of the software, controller, and UI. The methods and results from a small study on healthy subjects are presented, with questionnaire answers regarding the subjective experience of participants, along with quantitative performance metrics. 

\item Chapter 5 concludes with a summary of contributions and recommendations for future work. This includes features that were required which have not yet been implemented due to time constraints, and ideas for future direction of the project. 


\end{itemize}

	
\bibliography{thesis}
\bibliographystyle{ieeetr}
\end{document}